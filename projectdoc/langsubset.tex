\newcommand{\q}{\langle query\rangle}
\newcommand{\db}{\langle db\rangle}
\newcommand{\pat}{\langle pat\rangle}
\newcommand{\bug}{\langle bug\rangle}
\newcommand{\dist}{\langle distance\rangle}
\newcommand{\sem}[1]{\llbracket #1\rrbracket}
\newcommand{\lit}[1]{\texttt{#1}}

\newcommand{\column}{\langle column\rangle}
\newcommand{\dbtable}{\langle table\rangle}
\newcommand{\cond}{\langle cond\rangle}
\newcommand{\op}{\langle op\rangle}
\newcommand{\e}{\langle expr\rangle}
\newcommand{\ce}{\langle cexpr\rangle}

\begin{figure}[t]
%\scriptsize{%
\footnotesize%
\begin{align*}
\q ::= {} 
	& \texttt{ SELECT } \e^+ \texttt{ FROM } \dbtable^+ \\
        & \texttt{ WHERE } \cond^+ \\ 
	&  \texttt{ GROUP BY } \column^+ \texttt{ HAVING } \cond^+\\
\dbtable::= {} &\ atom \\
\column ::= {} &\ \dbtable.atom\\
\cond ::= {} &\ \ \cond \;\texttt{\&\&}\; \cond \\ 
    & |\ \cond \;\texttt{||}\; \cond \\
    & |\ \texttt{(}\;\cond\;\texttt{)} \\
    & |\ \ce \;\op\; \ce \\
\op ::= {} &\ \ \texttt{=} \;\;|\;\; \texttt{>}  \;\;|\;\; \texttt{<}\\
\ce ::= {} &\ \ const \;\;|\;\; \column  \;\; \\
\e ::= {} & \ce \;\;|\ count(\column) \\
    & |\ sum(\column) \;\;|\ max(\column) \;\;|\ min(\column) 
\end{align*}
\normalsize%
\caption{Syntax of the supported SQL subset.}
\label{fig:syntax}
\end{figure}


\section{Language and Example}
\label{sec:langsubset}

In this section, we first present the supported SQL subset, and
then describe a motivating example which could be expressed in the
supported SQL subset.

\subsection{Supported SQL Subset}

Figure~\ref{fig:syntax} defines the syntax of the supported language,
which
is a subset of the standard SQL 93 language. This language subset
supports common query operations across multiple tables (i.e.,
\CodeIn{select} ... \CodeIn{from} ... \CodeIn{where}..), and 
conjunction of predicates. The language
subset shares the same semantics with the standard SQL language.
Differing from language subsets used in the
existing work~\cite{DasSarma:2010}, our language subset
supporting
joining operations across multiple tables, and includes
widely-used database operations such as \CodeIn{group by},
\CodeIn{order by}, and \CodeIn{having}, as
well as a few common aggregation functions such as \CodeIn{count}, \CodeIn{sum},
\CodeIn{max}, and \CodeIn{min}.

As we will show in Section~\ref{sec:evaluation}, this
subset, though far from completion, is expressive enough for answering
many evaluated SQL exercises from a textbook and some real problems
from online forums.

\subsection{Motivating Example}
\label{sec:example}

%\section{A Motivating Example}
%\label{sec:example}

Consider the following SQL question picked up from a classic
database textbook~\cite{cowbook}: \textit{given a \CodeIn{student} table (Figure~\ref{tbl:student})
and an \CodeIn{enrolled} table (Figure~\ref{tbl:enrolled}), find out the name and max score of the
students whose level is senior and enrolled in more than 3 courses}.



\begin{figure}[t]
	\centering
\begin{tabular}{|c|c|c|}
\hline
 Student\_key& Student\_name & Level\\
\hline
 0001 & Adam & senior \\
 \hline
 0002 & Bob & junior \\
 \hline
 0003 & Erin & senior \\
 \hline
 0004 & Rob & junior\\
 \hline
 0005 & Dan & senior \\
 \hline
 0006 & Peter & senior \\
 \hline
 0007 & Sai & senior \\
 \hline
\end{tabular}
	\caption{An example input table `student'
for the SQL question described in
Section~\ref{sec:example}.}
	\label{tbl:student}
\end{figure}


\begin{figure}[t]
	\centering
\begin{tabular}{|c|c|c|}
\hline
 Student\_key& Course\_key& Score \\
\hline
 0001 & 001 & 4 \\
\hline
 0001 & 002 & 2 \\
\hline
 0002 & 001 & 3 \\
\hline
 0002 & 002 & 2 \\
\hline
 0002 & 003 & 3 \\
\hline
 0003 & 002 & 1 \\
\hline
 0004 & 001 & 4 \\
\hline
 0004 & 003 & 4 \\
\hline
 0005 & 002 & 5 \\
\hline
 0005 & 003 & 2 \\
\hline
 0005 & 004 & 1 \\
\hline
 0006 & 002 & 4 \\
\hline
 0006 & 004 & 5 \\
\hline
 0007 & 001 & 2 \\
\hline
 0007 & 003 & 3 \\
\hline
 0007 & 004 & 4 \\
 \hline
\end{tabular}
	\caption{An example input table  `enrolled'
for the SQL question described in
Section~\ref{sec:example}.}
	\label{tbl:enrolled}
\end{figure}	

\begin{figure}[t]
	\centering
\begin{tabular}{|c|c|}
\hline
 Student\_name & Max\_score \\
\hline
 Dan & 5 \\
\hline
 Sai & 5 \\
 \hline
\end{tabular}
	\caption{An example output table for the SQL question described in
Section~\ref{sec:example}.}
	\label{tbl:output}
\vspace{3mm}
\end{figure}		


\begin{figure}[t]
\begin{CodeOut}
\begin{alltt}
\textbf{select student.Student\_name, max(enrolled.Score)
from student, enrolled
where student.Student\_key = enrolled.Student\_key
      and student.Level = `senior'
group by student.Student\_key
having count(enrolled.Course\_key) > 3}
\end{alltt}
\end{CodeOut}
\vspace{-5mm}
	\caption{A SQL query to solve the SQL question described in
Section~\ref{sec:example}. When applied to the input tables in
Figure~\ref{tbl:student} and Figure~\ref{tbl:enrolled}, it produces the result table
in Figure~\ref{tbl:output}.}
	\label{fig:expected_sql}
\end{figure}

The question's description is quite simple.
For a novice user, although they have a clear
intention of what the query should do, the answer (Figure~\ref{fig:expected_sql}) may
not be that straightforward. 

Despite the possible difficult in writing a correct SQL query,
a user could still easily draw
two input tables (Figure~\ref{tbl:student} and Figure~\ref{tbl:enrolled})
and one output table (Figure~\ref{tbl:output}) that fulfill the
SQL question.

In the \CodeIn{student} table, column {\CodeIn{Student\_key}} with
\CodeIn{String} type serves as the primary key. 
%Columns {\CodeIn{Student\_name}} and {\CodeIn{Level}} are \textsf{String}-type.
In the \CodeIn{enrolled} table, both columns {\CodeIn{Student\_key}} and
{\CodeIn{Course\_key}} are two foreign keys, and column {\CodeIn{Score}}
with \CodeIn{Integer} type keeps students' scores on their enrolled courses.

In the output table, the first column  {\CodeIn{Student\_name}}
comes from table \textit{student}, and the second column {\CodeIn{Max\_score}}
is a aggregation attribute.% summarizing the with type of \CodeIn{Integer}.

Having the input and output examples, our technique successfully synthesizes
the desirable SQL query as shown in Figure~\ref{fig:expected_sql}.

