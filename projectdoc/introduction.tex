\section{Introduction}
\label{sec:introduction}


The big data revolution over the past few years has resulted
in significant advances in digitization of massive amounts
of data and accessibility of computational devices to massive
proportions of the population. A perennial challenge faced by many
enterprise nowadays is the management
of their increasingly large and complex databases, which can contain
hundreds and even thousands of tables. 


%but they often need to
%query databases to obtain relevant data to support their business decisions.


\subsection{End-users' Difficulties in Using SQL}
Although the relational database management system (RDBMS) and the
de facto language (SQL) are perfectly adequate for many end-users'
needs~\cite{Howe:2011}, the costs associated with deployment and
use of database software and SQL are prohibitive. 

The problem is exacerbated by the fact that many end-users
have myriad diverse backgrounds including research scientists,
business analysts, commodity traders, human resource managers,
finance professionals, and marketing managers. Increasingly,
those end-users who need to query databases to analyze the
data are not professional programmers, but are experts in some
other domains. They need to ask a variety of questions on their
data and use the answer to support their business decisions.
For example, as pointed out
by~\cite{Gray:2005}, conventional RDBMS software remains underused
in the long tail of science: the large number of users, such as the
research scientists who are in relatively small labs and individual
researchers, have limited IT funding, staff and infrastructure yet
collectively produce the bulk of scientific knowledge. 

%Similarly, other end-users
%such as business analysts who query databases frequently often
%do not have significant SQL expertise.

For a large number of end-users, they learn and use SQL queries in
the following typical ways: they browse textbook or online resources to learn basic
idioms of SQL. They try to write experimental SQL queries and execute them
on sample databases to explore the data itself. They modify
the written queries to derive new queries by adding or removing snippets:
predicates in the \CodeIn{where} clause,
tables in the \CodeIn{from} clause, or columns in the \CodeIn{select} clauses.
However, such practice is inefficient and time-consuming.
A key challenge is that many end-users can clearly understand
their goals but simply can not write a correct SQL query for
their tasks, either due to the syntax complexity of the language itself,
or the structure complexity of the underlying databases, or others.
For many end-users, they need an \textit{intelligent} tool
to assist them to perform database query tasks. A highly accessible tool
that end-users can describe their needs, and use to connect their
intentions to executable
SQL queries would be best.


\subsection{Existing Solutions}


\textit{Graphical User Interface} (GUI) and \textit{general programming languages}
are two state-of-the-art approaches in helping end-users perform
database queries. However, both of them are far from satisfactory.

Many RDBMS come with a well-designed GUI with tons of features.
However, 
%as a database management system often comes with tons
%of features, end-users struggle to find the correct feature or
%succession of commands to use from a maze of features to accomplish
%their tasks.
a GUI often does not permit users to personalize
a database's functionality for user-specific tasks. On the other hand,
as a GUI supports more and more customization features, users
may struggle to discover those features, which can significantly
degrade its usability. 

General programming languages, such as SQL,
Java (with JDBC), or other domain specific languages, 
serve as a fully expressive medium  for
communicating a user's intention to a database. However, learning
a practical programming language (even a simplified, high-level domain
specific language) often requires a substantial amount
of time and energy that a typical end-user would not prefer,
and should not be expected, to invest. 



\subsection{Synthesizing SQL Queries from Examples}

After carefully studying how end-users were describing their
encountered SQL query problems on online help forums, we observed that
one of the most common ways for end-users to
express their intents is using input-output examples. Although
input-output examples may lead to underspecification, they
serve as a straightforward way of describing \textit{what} the
task is and a natural interface that a tool can provide assistance.


In this work, we present a technique for synthesizing SQL queries
from input-output examples. In particular, our technique takes example input
table(s) and output table from the end-users, and then automatically
infers a SQL query (or multiple queries, if exist) that queries
the input database and returns the output example. If the inferred
SQL query is applied
to the example input, then it produces the example output, and if the
SQL query is applied to other similar inputs (potentially much larger tables),
then the SQL query produces a corresponding output.

Our technique follows a general methodology for designing
systems supporting programming by examples~\cite{Harris:2011}.
It includes the following three major steps:
(1) identify a language subset (of SQL) that is expressive to
describe a large proportion of %operations that users need to perform
database queries; (2) design an algorithm that
can efficiently infer programs in the language from input-output
examples; and (3) develop a ranking strategy that presents
the most likely program to the users, when multiple solutions exist.

\vspace{1mm}
\noindent \textbf{\textit{Supported SQL Subset.}}
The full SQL 93 specification contains over 1000 pages, and it is
infeasible and unnecessary to support all language features in practice.
When selecting a supported SQL subset, there is a tradeoff
between the expressiveness of the selected language subset,
and the complexity of finding simple consistent solutions
within that language's search space.  In general, the more expressive a search
space, the harder the task of finding consistent hypotheses within
that search space. 

During our study of end-user's real problems, we also carefully studied what would
be the solution to users' desired query, and found that the most common
desired queries can be composed by a handful of basic SQL constructs,
which form the target language of this work (Section~\ref{sec:langsubset}).

As shown in our experiments (Section~\ref{sec:evaluation}), the supported subset, though far from completion, is expressive
enough to describe many query tasks succinctly, while at the same time concise
enough to be amenable for efficient inference. 



\vspace{1mm}
\noindent \textbf{\textit{SQL Query Inference Algorithm.}} Our inference
algorithm, as described in Section~\ref{sec:approach}, consists of two
major steps. The first step, called \textit{SQL Skeleton Creation}, scans
the example input and output, and \textit{guesses} what a desirable
SQL query may look like. The output of this step is a set of \textit{incomplete} SQL
query skeletons. A SQL query skeleton is an incomplete SQL query,
which contains unknown parts. The second step,
called \textit{SQL Query Completion} takes as inputs a created
SQL skeleton, fills unknown parts with possible solutions, and then produces a set of
syntactically-valid SQL queries. In particular, when performing
SQL query completion, our technique uses a rule learning algorithm
to infer the conditions (Section~\ref{sec:decision_tree}), and uses 
type-directed search to infer possible aggregates in a query (Section~\ref{sec:agg_search}).

\vspace{1mm}
\noindent \textbf{\textit{Query Candidate Ranking Strategy.}}
Input-output examples provided by end-users are likely to
be under-specified. It is entirely possible that multiple
queries can be synthesized to satisfy the provided examples.
We address this problem by developing a ranking scheme that 
ranks the possibly multiple queries
consistent with the given input-output
examples. Our ranking scheme, detailed in Section~\ref{sec:ranking}
is inspired by the Occam's razor principle, prefers a smaller and
simpler solution if other aspects being equal.

%is usually the correct one. 

%Besides
%selecting a smaller one, we could also compute the likelihood
%of each query being correct based on some statistics, and then
%pick the most likely one.


\subsection{Technique Evaluation}

The goal of this work is developing a practical SQL query synthesis technique that is capable of
helping end-users writing a wide range of SQL queries.
The technique aims to replace the role of the forum expert,
which not only removes human from the loop, but also enables
end-users to solve their problems in a few seconds as opposed to a few days
(while waiting for an expert's reply). 

Thus, we implemented our technique in an open-source tool
and evaluated it in two ways. First, we picked up 6 SQL
exercises from a classic database textbook~\cite{cowbook}. SQL exercises
from a database textbook often cover
the most widely-used SQL features 
that a course instructor wishes students to master.
Those exercises can be served as golden test
to evaluate the expressiveness of our supported SQL subset. Second,
we collected 5 non-trivial SQL problems raised by real end-users on online
help forums, and tested whether our technique can effectively synthesize desirable
SQL queries.

As a result, our technique successfully synthesized 5 out of 6
textbook exercises and solved all 5 forum problems, within a very
small amount of time (1 minute per exercise/problem, on average).

\subsection{Contributions}

This paper makes the following contributions:

\begin{itemize}
\item \textbf{Problem.} To the best of our knowledge, we are the first
to address the SQL query synthesis problem from examples across multiple
tables.
%An approach to automatically synthesizing SQL
%queries from input-output examples (Section~\ref{sec:approach}).

\item \textbf{Technique.} We show to automatically synthesize SQL
queries from input-output examples (Section~\ref{sec:approach}).

\item \textbf{Tool.} A practical tool that implements the proposed technique (available at:
\url{http://sqlsynthesizer.googlecode.com}).

\item \textbf{Evaluation.} An empirical evaluation of the tool implementation
on a number of SQL exercises from a classic textbook~\cite{cowbook}
and SQL problems raised by real users on online forums (Section~\ref{sec:evaluation}).
\end{itemize}
